\documentclass[floats,floatfix,showpacs,amssymb,prd,twocolumn,superscriptaddress,nofootinbib]{revtex4-1}

\usepackage{amssymb,amsmath,verbatim,color,mathtools,needspace,enumitem,etoolbox,xcolor}
\usepackage{amsmath,color,graphicx,fancyvrb}

\usepackage[normalem]{ulem}
\usepackage{hyperref}
\definecolor{linkcolor}{rgb}{0.0,0.3,0.5}
\hypersetup{colorlinks=true,linkcolor=linkcolor,citecolor=linkcolor,filecolor=linkcolor,urlcolor=linkcolor}
\newcommand\prlsec[1]{\vspace{2mm}\noindent \emph{#1}--}
\usepackage{xspace}
\usepackage{hologo}
\usepackage[T1]{fontenc}


\def\aj{\rm{AJ}}
\def\apj{\rm{ApJ}}
\def\apjl{\rm{ApJ}}
\def\pasj{PASJ}
\def\apjs{\rm{ApJS}}
\def\mnras{\rm{MNRAS}}
\def\prd{\rm{PRD}}
\def\prl{\rm{PRL}}
\def\prx{\rm{PRX}}
\def\cqg{\rm{CQG}}
\def\araa{\rm{ARA\&A}}
\def\nat{\rm{Nature}}
\def\aap{\rm{A\&A}}
\def\physrep{\rm{Phys.~Rep.}}
\newcommand{\filltex}{\texttt{filltex}\xspace}
\newcommand{\latex}{\LaTeX\xspace}
\newcommand{\tex}{\TeX\xspace}

\newcommand{\bibtex}{\hologo{BibTeX}\xspace}
\newcommand\GitHub{\textsf{GitHub}\xspace}

\makeatletter
\preto{\@verbatim}{\topsep=1pt \partopsep=4pt }
\makeatother

\makeatletter
\def\preparefootins{%
\global\rcol@footinsskip\skip\footins
\global\skip\footins\z@
\global\count\footins\z@
\global\dimen\footins2\textheight}
\makeatother


\begin{document}

\title{\texttt{filltex}: Automatic queries to ADS and INSPIRE databases to fill \latex bibliography
}


\author{Davide Gerosa}
\thanks{Einstein Fellow}
\email{dgerosa@caltech.edu}
\affiliation{TAPIR 350-17, California Institute of Technology, 1200 E California
Boulevard, Pasadena, CA 91125, USA}
\author{Michele Vallisneri}
%\email{michele.vallisneri@jpl.nasa.gov}
\affiliation{TAPIR 350-17, California Institute of Technology, 1200 E California
Boulevard, Pasadena, CA 91125, USA}
\affiliation{Jet Propulsion Laboratory, California Institute of Technology, 4800 Oak Grove Drive,
Pasadena, CA 91109, USA}





\begin{abstract}
\filltex is a simple tool to fill \latex reference lists with records from the ADS and INSPIRE databases. ADS and INSPIRE are the most common databases used among the astronomy and theoretical physics  communities, respectively. \filltex automatically looks for all citation labels present in a \tex document and, by means of web-scraping, downloads all the required citation records from either of the two databases. \filltex significantly speeds up the \latex scientific writing workflow, as all required actions (compile the \tex file, fill the bibliography, compile the bibliography, compile the \tex file again) are automated in a single command. We also provide an integration of \filltex for the macOS \latex editor TexShop.
\end{abstract}


\maketitle



\latex is a typesetting system widely used to prepare scientific publications (\href{https://www.latex-project.org}{latex-project.org}). Within the \latex ecosystem, bibliography lists are handled by the \bibtex software (\href{http://www.bibtex.org/}{bibtex.org}).
When a \tex document is compiled, an auxiliary file (with extension \texttt{.aux}) is first created with information of all references inserted with the  \verb|\cite| command. If these entries are available on a specified bibliography file (\texttt{.bib} extension), \bibtex parses them into a further auxiliary file (\texttt{.bbl} extension). Two additional \latex executions generate the final document with all bibliographic items in place.

Extended databases of academic publications with up-to-date bibliographic information are available online. In particular, the most widely used database among the astronomy community is the Astrophysics Data System (ADS, \href{http://adsabs.harvard.edu/}{adsabs.harvard.edu}), operated for NASA by the Smithsonian Astrophysical Observatory. The high energy and theoretical physics communities widely utilize the INSPIRE database (\href{http://inspirehep.net/}{inspirehep.net}), which combines the earlier Stanford Physics Information Retrieval system with the Invenio digital library framework, and is operated by a collaboration of CERN, DESY, Fermilab, IHEP, and SLAC. Both databases provide reference records in \bibtex format.

Here we present v1.2 of \filltex, a \texttt{python}-based web-scraping tool to  fill \latex  bibliography files with records from ADS and INSPIRE. The code  first detects all the citation keys from an input \tex file. Such keys are then matched to entries from either of the two online databases and used to download full bibliographic records. Finally, the output document is generated by calling the \texttt{bibtex} and \texttt{pdflatex} commands appropriately. \filltex allows for a mixture of keys from ADS and INSPIRE, possibly in combination with other records introduced manually with personalized keys.

\filltex is publicly available on GitHub at \href{https://github.com/dgerosa/filltex.git}{github.com/dgerosa/filltex}, where more extensive  documentation is presented (including e.g. handling of  arXiv pre-prints and journal abbreviations). 
Installation is straightforward:
\vspace{-0.1cm}
%\begin{Verbatim}[fontsize=\normalsize,commandchars=\\\[\]]
%\color[linkcolor]git clone https://github.com/dgerosa/filltex.git
%\color[linkcolor]cd filltex
%\color[linkcolor]chmod +x bin/*
%\color[linkcolor]echo "PATH=$PATH:$(pwd)/bin" >> ${HOME}/.bashrc
%\color[linkcolor]source ${HOME}/.bashrc
%\end{Verbatim}
%\begin{Verbatim}[fontsize=\normalsize,commandchars=\\\[\]]
%\color[linkcolor]git clone https://github.com/dgerosa/filltex.git
%\color[linkcolor]cd filltex
%\color[linkcolor]python setup.py install
%\end{Verbatim}
\begin{Verbatim}[fontsize=\normalsize,commandchars=\\\[\]]
\color[linkcolor]pip install filltex
\end{Verbatim}
\vspace{-0.1cm}

The code can be executed by typing, e.g.:
\vspace{-0.1cm}
\begin{Verbatim}[fontsize=\normalsize,commandchars=\\\[\]]
\color[linkcolor]filltex example
\end{Verbatim}
\vspace{-0.1cm}
The command above compiles a \tex file \texttt{example.tex} into  a final document \texttt{example.pdf}, with all downloaded references in place. The name of the \texttt{.bib} file containing the downloaded records is guessed from the \verb|\bibliography| command specified in \texttt{example.tex}. Example files are available in the GitHub repository.

We also provide an implementation of \filltex for the macOS \latex editor TexShop (\href{http://pages.uoregon.edu/koch/texshop/}{pages.uoregon.edu/koch/texshop}). The \filltex  engine for TexShop can be installed with:
\vspace{-0.1cm}
%\begin{Verbatim}[fontsize=\normalsize,commandchars=\\\[\]]
%\color[linkcolor]cp filltex.engine
%\color[linkcolor]    ~/Library/TeXshop/Engines/filltex.engine
%\end{Verbatim}
\begin{Verbatim}[fontsize=\normalsize,commandchars=\\\[\]]
\color[linkcolor]filltex install-texshop
\end{Verbatim}
\vspace{-0.1cm}
After restarting TexShop, \filltex will be available in the ``Typeset'' dropdown menu among other compiling options.

Future improvements include support for other databases, such as Google Scholar, and personalized journal abbreviations. Compatibilities with other services, such as Overleaf, Authorea and Mendeley, may also be investigated. Contributors are welcome!


\vspace{0.2cm}
\textbf{{Acknowledgments.}}
This project started from the \texttt{python} course taught by M.V. at Caltech in 2012 (\href{http://www.vallis.org/salon/}{vallis.org/salon}). We thank Lars Holm Nielsen, reviewer for The Journal of Open Software. We thank all developers, contributors and supporters of \latex, \bibtex, ADS, INSPIRE, GitHub and TexShop. D.G. is supported by NASA through Einstein Postdoctoral Fellowship Grant No. PF6-170152 awarded by the Chandra X-ray Center, which is operated by the Smithsonian Astrophysical Observatory for NASA under Contract NAS8-03060. 

%\bibliography{filltex_arxiv}

\end{document}
