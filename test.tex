\documentclass[floats,floatfix,showpacs,amssymb,twocolumn,superscriptaddress,nofootinbib]{revtex4}

\usepackage{aas_macros}

\begin{document}

\title{davide\_tex.sh: efficient latex workflow}

\author{Davide Gerosa}
\email{d.gerosa@damtp.cam.ac.uk}
\affiliation{Department of Applied Mathematics and Theoretical Physics, Centre for Mathematical Sciences, University of Cambridge, Wilberforce Road, Cambridge CB3 0WA, UK}

\date{\today}

\begin{abstract}

This is meant to document and test my script to improve my Latex workflow.
\end{abstract}
\maketitle 

\cite{2015EPJC...75..212K}


\section{What it is}

This is little project to improve my Latex workflow. I did this mainly because I hate going to ADS or INSPIRE and manually cut\&paste the bibliographic records.

What happens when you compile a latex file? How's bibliography handled?

\begin{enumerate}
\item You first run latex and all requested citations get dumped into an .aux file
\item You \emph{should} have the required entries in you .bib file
\item Run bibtex, which looks for the citations inside the bib file, and writes the results into a .bbl
\item Run latex again, which processes the .bbl into the compiled paper, and dumps the citation handles into .aux again
\item Finally run latex again, which gets the correct citation handles into the compiler paper
\end{enumerate}

The commands you need to run are: pdflatex, bibtex, pdflatex, pdflatex. These, of course can be put into a bash script and done in one goal.
Here we also want to automatically solve the second point,  looking for citations on ADS (if you're an astronomer), INSPIRE (if you're a theoretical physicists) or both of them (if you do gravitational waves).
The main idea is to query the websites and create/update your .bib file with the records \emph{without} go on the websites and cut\&paste each record manually.

Python is an efficient way to do this, because of the \url{requests} package. Before that, you need a working python distribution to run these things. In my opinion, the best way to install python is using a virtual environment\footnote{\url{http://virtualenv.readthedocs.org/en/latest/virtualenv.html}}, because it comes with the installing tool \url{pip} where developers put their modules. Moreover it's virtual: if you mess things up, just delete the environment and create a new one  without screwing up your system. If you don't have \url{requests} already installed, try \url{pip~install~requests} first.

ADS and INSPIRE are quite different from each other, so I'll use two different python scripts to query the two databases. Credits go to Michele Vallisneri for the ADS script\footnote{http://vallis.org/salon/summary-2.html} and Ian Huston for the INSPIRE script\footnote{http://www.ianhuston.net/2012/06/pyinspire-python-script-to-access-inspire-database/}. I only touched them slightly and wrapped things together with a bash script \url{davide_tex.sh}.
ADS record typically contains journal abbreviations, defined here\footnote{\url{http://doc.adsabs.harvard.edu/abs_doc/aas_macros.sty}}. The macro is automatically download if needed.

At the end, I'm also running a perl script\footnote{\url{http://app.uio.no/ifi/texcount/}} to count the words in each section: useful to write letters, reports...


\section{Really?}
Place the script directory somewhere \url{<your_location>} in your system; open \url{dtex.sh} and modify the first line with your chosen location.

Open the terminal, go into a directory where you keep your paper and run 
\url{bash <your_location>/dtex.sh}


You can use this file to test my script. First of all, remove everything  you may already have (aux, bib, bbl)  but this tex file.

This is a citation from ADS: \url{\cite{1975ApJ...195L..51H}} \cite{1975ApJ...195L..51H}

This is a citation from INSPIRE:\url{\cite{Hulse:1974eb}} \cite{Hulse:1974eb}




\cite{2008PhRvL.101p1101S}

\section{Known limitations}
\begin{enumerate}

\item Of course, computers are stupid. You must enter the cite keys in your tex file using the standard format used by the databases. You can't use your nicknames \url{\cite{awesome_paper}} but you must keep \url{\cite{1975ApJ...195L..51H}} or \url{\cite{Hulse:1974eb}}. ArXiv number are also fine \url{\cite{ gr-qc/0610154 | }} but they will be changed in the tex file to match the INSPIRE entry

\item Currently, the name of the bibliography file must be  the same as the one of the tex file you're running. I could change this, but I never bothered, because I just use a different bib for each different paper.

\item eprints in ADS are tricky. When an print get published they change the cite key, but keep the url. This means that if you have a old arXiv reference, my script will like to store it with a different records. I fixed this now: this is a published paper \cite{2010PhRvD..81h4054K} and this is its arXiv version \cite{2010arXiv1002.2643K}. It will appear twice in the bibliography, that's unavoidable because the script can't know that the paper is the same if it appears with two different (both allowed!) keys. INSPIRE doesn't have this problem, because they don't change the cite key when a paper get published.
\item I'd like to write an engine to do all of this from TexShop, which I like. I tried, but I've been unsuccessful so far. I still need a way to pass parameters.  

\end{enumerate}

\bibliography{test2}
\end{document}


